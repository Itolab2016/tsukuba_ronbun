\section{ロボットの起動方法}
ロボットが起動するまでに必要な手順を以下に示す.
また,ロボットを起動する際に使用する操作パネルの写真を図\ref{fig:start_panel}に示す.さらに,起動用の外付けHDDを図\ref{fig:start_hdd}に示す.
\subsection{起動}
\begin{enumerate}
\item 操作パネルのうち左上の電源スイッチをonにする.
\item 起動用の外付けSSDが接続されていない場合のみキーボードを接続し「s」と入力する.
\item ロボットから「ubuntuが起動しました」という音声が聞こえるまで待機する.
\end{enumerate}
\subsection{無線通信}
\begin{enumerate}
\item 使用するデバイスをロボットが接続しているネットワークと同じネットワークに接続する.
\item sshコマンドを用いてロボットとデバイスを接続する.

(例 ssh ubuntu@192.168.2.116  ssh ユーザー名@ロボットのIPアドレス   
\item 接続が完了するとデバイス側のユーザー名がロボットのユーザー名に切り替わる.
\item 接続できない場合はロボットがネットワークに接続しているか確認する(ロボットのwifiドングルが機能しているかなど).  
\end{enumerate}
